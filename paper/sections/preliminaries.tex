\section{Preliminaries}\label{sec:preliminaries}

\subsection{Abstract Argumentation Frameworks}

We adopt the foundational model of \cite{dung1995acceptability} as the backbone of our verification and repair pipeline.

\begin{definition}[Abstract Argumentation Framework]\label{def:af}
An \emph{abstract argumentation framework} (AF) is a pair $F = (\mathcal{A}, \mathcal{R})$ where $\mathcal{A}$ is a finite set of arguments and $\mathcal{R} \subseteq \mathcal{A} \times \mathcal{A}$ is a binary attack relation. We write $a \rightsquigarrow b$ whenever $(a,b) \in \mathcal{R}$, meaning $a$ attacks $b$.
\end{definition}

\begin{example}[Continuing Example~\ref{ex:running}]\label{ex:af}
The initial AF is $F_0 = (\{a_0, a_1, a_2, a_3, a_4\}, \{(a_0, a_4), (a_3, a_0)\})$, where $a_0$~(``symptoms are non-specific'') attacks the target~$a_4$ and $a_3$~(``Lupus commonly presents with these symptoms'') counterattacks~$a_0$; after the negative ANA result, $F_1 = (\{a_0, a_1, \ldots, a_5\}, \{(a_0, a_4), (a_3, a_0), (a_5, a_3)\})$, as shown in Figure~\ref{fig:af-evolution}(a--b).
\end{example}

Intuitively, an argument is accepted if every objection against it can be countered; different semantics formalize this intuition with varying degrees of caution.
Given an AF $F = (\mathcal{A}, \mathcal{R})$, a set $S \subseteq \mathcal{A}$ is \emph{conflict-free} if no two arguments in $S$ attack each other. An argument $a$ is \emph{defended} by $S$ if every attacker of $a$ is attacked by some member of $S$. A conflict-free set $S$ is \emph{admissible} if it defends all its elements. The principal semantics we employ are the \emph{grounded} extension, which is the unique minimal complete extension obtained as the least fixed point of the characteristic function; the \emph{preferred} extensions, which are maximal admissible sets; and \emph{stable} extensions, which are conflict-free sets that attack every argument outside themselves~\cite{baroni2018handbook}. Throughout this paper we write $\sigma(F)$ to denote the set of extensions of $F$ under semantics $\sigma \in \{\mathit{gr}, \mathit{pr}, \mathit{st}\}$.

\subsection{Argumentation Semantics for Explanation}

We now define the key notion linking argumentation semantics to explanation.
An argument $a \in \mathcal{A}$ is \emph{credulously accepted} under $\sigma$ if $a$ belongs to at least one extension in $\sigma(F)$, and \emph{skeptically accepted} if it belongs to every extension.

\begin{definition}[Defense Set]\label{def:defense-set}
Given an AF $F = (\mathcal{A}, \mathcal{R})$, semantics $\sigma$, and a skeptically accepted argument $t \in \mathcal{A}$, a \emph{defense set} for $t$ is a minimal admissible set $D \subseteq \mathcal{A}$ such that $t \in D$. We write $\mathit{Def}_\sigma(t)$ for the collection of all defense sets of $t$ under $\sigma$.
\end{definition}

\begin{example}[Continuing Example~\ref{ex:running}]\label{ex:defense}
In~$F_0$ (Figure~\ref{fig:af-evolution}a), $D = \{a_3, a_4\}$ is a defense set for~$a_4$: it is conflict-free, $a_3$ defends~$a_4$ by attacking~$a_0$, and $D$~is minimal since removing~$a_3$ would leave $a_4$ undefended against~$a_0$.
In~$F_1$, $D$~is no longer admissible because $a_3$ is attacked by~$a_5$ with no counterattack, so the defense of~$a_4$ collapses.
\end{example}

Defense sets serve as formal explanations: each $D \in \mathit{Def}_\sigma(t)$ identifies the smallest self-defending coalition that sustains $t$, transforming opaque LLM rationales into objects whose validity can be checked against argumentation semantics~\cite{dunne2009complexity}.

\subsection{Task Setting}

We consider a setting in which an LLM receives a question $q$ and produces an answer $a$ with a free-form explanation $e$, which \textsc{Argus} transforms into a formal argumentation structure.

\begin{definition}[Explanation Verification Task]\label{def:task}
Given a question $q$, an LLM-generated answer $a$, and an explanation $e$, the \emph{explanation verification task} produces a tuple $(G, v, \rho)$ where $G = (\mathcal{A}, \mathcal{R})$ is an argument graph constructed from $e$, $v \in \{\mathit{accepted}, \mathit{rejected}, \mathit{undecided}\}$ is the verification verdict for the target argument $a_t$ representing $a$ under semantics $\sigma$, and $\rho$ is an optional repair operator applied when $v \neq \mathit{accepted}$. An \emph{evidence update} $\Delta = (\mathcal{A}^+, \mathcal{R}^+, \mathcal{A}^-, \mathcal{R}^-)$ specifies new arguments and attacks to be added or removed, reflecting newly available facts or counterarguments.
\end{definition}

\begin{example}[Continuing Example~\ref{ex:running}]\label{ex:verify}
In~$F_0$, the verification task produces $v = \mathit{accepted}$ for~$a_4$ under grounded semantics: $a_3$ defeats the differential~$a_0$, so the grounded extension is $\{a_1, a_2, a_3, a_4\}$.
After incorporating the evidence update $\Delta = (\{a_5\}, \{(a_5, a_3)\}, \emptyset, \emptyset)$, $a_5$ defeats~$a_3$, reinstating~$a_0$, and the verdict becomes $v = \mathit{rejected}$, triggering the repair operator~$\rho$.
\end{example}

The target $a_t$ is \emph{accepted} under $\sigma$ if it belongs to at least one $\sigma$-extension (credulous acceptance), and \emph{rejected} if it belongs to no extension. Under grounded semantics, an argument may also be \emph{undecided}---belonging to no extension yet not attacked by the grounded extension---and credulous and skeptical acceptance coincide.

\subsection{Explanation Repair Problem}

When an evidence update $\Delta$ renders the explanation inconsistent, the system must revise the argument graph following the principle of minimal change~\cite{alchourron1985agm}.

\begin{definition}[Minimal-Change Repair Problem]\label{def:repair}
Let $\mathit{AF} = (\mathcal{A}, \mathcal{R})$ be an AF, $\sigma$ a semantics, $a_t \in \mathcal{A}$ a target argument, $s \in \{\textsc{in}, \textsc{out}\}$ a desired status, $\Delta$ an evidence update, and $\kappa$ a strictly positive cost function ($\kappa(o) > 0$ for every operation~$o$). A \emph{repair} is a sequence of edit operations $\mathit{Ops} = \langle o_1, \ldots, o_m \rangle$ where each $o_i$ is one of $\mathsf{add\_arg}(a)$, $\mathsf{del\_arg}(a)$, $\mathsf{add\_att}(a,b)$, or $\mathsf{del\_att}(a,b)$ for arguments $a, b$. Let $\mathit{AF}' = \mathsf{apply}(\mathit{AF}, \Delta, \mathit{Ops})$ denote the framework obtained by first incorporating $\Delta$ and then executing $\mathit{Ops}$. A repair is \emph{valid} if $a_t$ has status $s$ under $\sigma$ in $\mathit{AF}'$, and an \emph{optimal repair} minimizes $\sum_{i=1}^{m} \kappa(o_i)$ over all valid repairs.
\end{definition}

\begin{example}[Continuing Example~\ref{ex:running}]\label{ex:repair-ex}
As shown in Figure~\ref{fig:af-evolution}(c), the repair $\mathit{Ops} = \langle \mathsf{add\_arg}(a_6), \mathsf{add\_att}(a_6, a_5) \rangle$ restores~$a_4$ at total cost~$2$ under uniform cost ($\kappa \equiv 1$).
The alternative $\mathit{Ops}' = \langle \mathsf{del\_arg}(a_5) \rangle$ costs~$1$ but discards evidence; under structure-preserving cost with $\kappa(\mathsf{del\_\cdot}) = 2\kappa(\mathsf{add\_\cdot})$, both repairs cost~$2$, and domain preferences break the tie.
\end{example}

The cost function $\kappa$ encodes domain-specific preferences (e.g., deletions costlier than additions), connecting to enforcement in abstract argumentation~\cite{baumann2010complexity,cayrol2019argumentation} while adding an explicit cost model for explanation maintenance.

