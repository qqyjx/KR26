% ===== §6  Related Work =====
\section{Related Work}\label{sec:related}

Our work connects three lines of research: argumentation-based approaches to LLM reasoning, self-correction methods for language models, and formal theories of belief change in argumentation.

\textbf{Argumentation and LLMs.}
Vassiliades et al.~\cite{vassiliades2021argumentation} survey argumentation for explainable AI; our work instantiates this vision with a concrete repair operator for LLM self-explanations.
ArgLLMs~\cite{freedman2025arglm} constructs Dung-style graphs from LLM claims but treats verification as terminal, with no update mechanism.
ARGORA~\cite{argora2026} orchestrates multi-agent argumentation-mediated dialogue with causal semantics, but its correction operates through re-deliberation rather than a formally defined repair operator.
MQArgEng~\cite{mqargeng2024} demonstrates that modular argumentation engines improve LLM reasoning but does not address explanation maintenance.
\textsc{Argus} differs from all three by providing a minimal-change repair operator with AGM-compliant guarantees.
Bengel and Thimm~\cite{bengel2025sequence} introduce \emph{sequence explanations} tracing why arguments are accepted; \textsc{Argus} addresses the dual question of \emph{how to restore} acceptance, and the two could be composed.
We adopt Dung-style abstract argumentation rather than ASPIC$^+$~\cite{modgil2014aspic} because the complexity bounds we exploit (Theorem~\ref{thm:complexity}) are established for this setting.

\textbf{Self-Correction and Revision.}
Self-Refine~\cite{madaan2023selfrefine} and Reflexion~\cite{shinn2023reflexion} iteratively rewrite LLM outputs but without formal minimality guarantees---previously correct reasoning may be silently discarded.
Huang et al.~\cite{huang2024selfcorrect} demonstrate that LLMs cannot self-correct without external feedback.
RARR~\cite{gao2023rarr} retrieves evidence for revision but targets surface-level attribution; SelfCheckGPT~\cite{manakul2023selfcheckgpt} detects hallucinations but provides no repair mechanism.
Chain-of-Verification~\cite{dhuliawala2024cove} and CRITIC~\cite{gou2024critic} improve factual accuracy but lack formal preservation guarantees.
Matton et al.~\cite{matton2025walk} measure faithfulness through counterfactual interventions and Bayesian causal models, demonstrating that instruction-tuned models produce more faithful explanations; our evaluation applies counterfactual probes to argumentation-structured explanations to verify repair correctness.
In contrast, \textsc{Argus} formalizes the repair search space as edits to an argumentation framework, bounds the cost of change, and guarantees that unaffected reasoning steps are preserved.

\textbf{Belief Revision and Argumentation Dynamics.}
The AGM theory~\cite{alchourron1985agm} and the revision/update distinction~\cite{katsuno1992update} provide the classical foundations for principled belief change.
Hase et al.~\cite{hase2024fundamental} argue that model editing in LLMs is fundamentally a belief revision problem and identify challenges in applying AGM rationality criteria to neural knowledge stores; our work sidesteps these challenges by operating on an \emph{external} argumentation structure rather than on model parameters, making the AGM postulates directly applicable.
Our evidence update~$\Delta$ is closer to the Katsuno--Mendelzon notion of \emph{update}---adapting beliefs to a changed world---than to \emph{revision}, which incorporates new information about a static world, since each~$\Delta$ reflects genuinely new evidence rather than a correction of prior beliefs. Nonetheless, we adopt the AGM postulates as rationality criteria because the minimal-change desiderata they formalize are independent of this distinction.
In argumentation, Cayrol et al.~\cite{cayrol2020argumentation} and Baumann and Brewka~\cite{baumann2010complexity} study how structural modifications affect extensions and the complexity of enforcement; Coste-Marquis et al.~\cite{costemarquis2014enforcement}, Wallner et al.~\cite{wallner2017complexity}, and Bisquert et al.~\cite{bisquert2013repair} formalize argumentation revision as minimal status or structural change.
Mailly~\cite{mailly2024constrained} extends enforcement to constrained incomplete argumentation frameworks, where structural constraints limit the set of completions available for reasoning; our $k$-neighborhood approximation similarly constrains the search space, though we target a different problem---repair under evidence updates rather than enforcement under uncertainty.
Alfano et al.~\cite{alfano2024counterfactual} develop counterfactual explanations for abstract argumentation via weak-constrained ASP encodings; their approach identifies minimal changes that would \emph{reverse} an acceptance verdict, whereas \textsc{Argus} computes minimal changes that \emph{restore} a verdict disrupted by external evidence.
In particular, Coste-Marquis et al.\ enforce a desired extension through minimum structural modifications, whereas our formulation targets a single argument's status, incorporates evidence updates as a first-class input, and supports heterogeneous cost functions that reflect argument-level confidence.
\textsc{Argus} extends these foundations to LLM explanation maintenance, introducing a weighted cost model tailored to argument confidence and structural role.
We now formalize the core concepts underlying the \textsc{Argus} framework.

