% ===== §7  Conclusion =====
\enlargethispage{3\baselineskip}
\section{Conclusion}\label{sec:conclusion}

We presented \textsc{Argus}, a framework that structures LLM self-explanations as argumentation frameworks, verifies them against formal semantics, and repairs them at minimum cost when new evidence arrives.
The minimal-change repair operator satisfies adapted AGM postulates---success, inclusion, and vacuity---and a representation theorem shows that these three postulates \emph{bidirectionally characterize} the class of minimum-cost repair operators under positive costs, providing formal guarantees absent from existing approaches.
Theoretically, the repair problem is tractable under grounded semantics, NP-complete under preferred and stable semantics, and $\Sigma_2^P$-complete under skeptical stable semantics; the $k$-neighborhood approximation maintains scalability in practice.
Experiments on HotpotQA and FEVER yielded relative improvements of \improveFaithfulness{} in faithfulness and \improveContestability{} in contestability over the strongest argumentation baseline, while achieving the lowest repair cost among all repair-capable methods.

Several limitations point to future work.
The framework's quality depends on the LLM's ability to decompose rationales into atomic units; extraction errors propagate through the pipeline, though the NLI and ASP verification stages mitigate this dependency.
While the $k$-neighborhood approximation handles the framework sizes encountered in our experiments, densely connected frameworks with hundreds of arguments may require more aggressive strategies.
The evaluation relies primarily on automatic metrics over fact-checking and multi-hop QA datasets; extending the approach to open-ended generation---where the target argument may lack a ground-truth referent---would require alternative acceptance criteria such as coherence-based semantics.
Looking ahead, composing \textsc{Argus} with sequence explanations~\cite{bengel2025sequence} would yield a bidirectional explanation infrastructure, and integrating the repair operator into retrieval-augmented generation pipelines could provide continual explanation maintenance as knowledge bases evolve, particularly in high-stakes domains where audit trails of explanation changes are mandated.

